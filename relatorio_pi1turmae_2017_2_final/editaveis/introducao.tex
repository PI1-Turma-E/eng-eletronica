\chapter{Introdução}

\section{ASPECTOS GERAIS}
O ultrassom é um som que possui uma frequência superior à que os humanos são capazes de ouvir. Os aparelho que utilizam o ultrassom são capazes de gerar de 20 kHz a giga-hertz, no caso deste estudo o aparelho em foco é o de ultrassonografia ou ecografia.
A ultrassonografia é um aparelho de diagnóstico médico que usa o eco de ondas ultrassônicas de alta frequência para visualizar o corpo internamente em tempo real. O diagnóstico que ele oferece é feito por imagens, entretanto essas imagens não são de grande qualidade quando comparadas a de outros aparelhos de diagnóstico, mas a sua grande vantagem é o preço, que é bem mais acessível e também é de rápida execução e não necessita de muito espaço. Essas característica são de extrema importância para o projeto.

\section{APARELHOS DE ULTRASONOGRAFIA}
Existem diversos tipos de aparelhos de ultrassonografia de diversos tamanhos e para diversos fins, mas basicamente eles são formados pelo computador que mostra a imagem e os transdutores. Há diversos tipos de transdutores, o uso de um tipo específico depende de qual local do corpo vai ser examinado, entre esses tipos estão: transdutores curvos, lineares, endocavitários, setoriais, especiais, etc. 

\section{RAZÃO DO PROJETO}
Há um grande número de diagnósticos que podem ser feitos através de exame de ultrassom, entretanto para esse exame é necessário pessoal especializado para realizá-lo no paciente, mas que não estão muito presentes fora de grandes centros urbanos, por esse fato surgiu a ideia do projeto, que quer levar esse exame para locais fora dos grandes centros a fim de atender a população que necessita desse exame, mas que não consegue acesso, pois não possui um profissional capaz de manusear o aparelho. Dessa forma, o projeto é encontrar uma forma, através da telemedicina, capaz de fazer com que o profissional não necessite estar presente na sala com o paciente, mas que possa controlar o exame à distância. Para isso, foi pensado um braço mecânico capaz de segurar o transdutor do aparelho de ecografia e realizar os movimentos desejados pelo profissional, que irá controlar esse braço por meio de um joystick, assim não há a necessidade da presença dele no mesmo local que o paciente.
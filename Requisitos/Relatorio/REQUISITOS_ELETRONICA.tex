\documentclass[a4paper]{article}

\usepackage[brazilian]{babel}
\usepackage[utf8]{inputenc}
\usepackage[T1]{fontenc}

\title{Requisitos Eletrônicos - Projeto Integrador 1 Turma E}
\author{Tiago Rodrigues Pereira}
\date{06/09/2017 - atualizado 16/09/2017} 

\begin{document}

\maketitle

\section{Objetivo}

\paragraph{} O objetivo deste documento é a definição dos requisitos básicos para o projeto relacionado à disciplina Projeto Integrador 1 - Turma E que envolve a criação de um sistema de ultrasom movimentado por um robô mecatrônico e controlado remotamente com dois sistemas: O sistema principal, onde o médico controla o robô, e o sistema escravo, onde existe um técnico que monta sistema de ultrasom robôtico e monitora o  decorrer do exame com ultrasom.

\paragraph{} Cada um dos requisitos eletrônicos abaixo foram levantados buscam apresentar aos membros da equipe uma visão geral do que deve ser pesquisado e podem ser alterados no decorrer da pesquisa (me avisem que altero o documento).

\section{Orientação}
\paragraph{} Primeiramente todos os documentos, artigos, sites utilizados, imagems, gráficos, sinais divinos, etc, façam download ou citem no documento resumo indicando de onde veio aquela informação ou qual o nome do pdf que veio aquelas informações. 

\paragraph{} Após as pesquisas realizadas façam um resumo do que foi pesquisado. Indique as vantagens, desvantagens, se é melhor ou pior que algum outro tipo de componente, porque , se existe atraso, qual é ele, ..., ou seja, que esteja relacionado ao tema que você está inserido e indiquem na bibliografia de onde veio aquela informação. 

\paragraph{} Ao final crie um zip com todos os arquivos pesquisados e o resumo em um pdf e façam upload no \textbf{Github} (se não for possível enviem no \textbf{Slack} no channel da eletrônica)

\paragraph{} Qualquer \textbf{dúvida} falem usando o \textbf{Slack} me enviando uma \textbf{mensagem privada} por lá.

\paragraph{Prazo:} Entrega máxima até o dia \textbf{14/09/2017} (Quinta - Feira)

\subparagraph{Observação 1:} Para evitar a criação de um efeito cascata, por favor, entregue até essa data pois ainda tenho que fazer um resumo de tudo (14 pessoas) e enviar para a gerente geral concatenar tudo e criar o nosso artigo final para o ponto de controle 1 que será entregue até o dia \textbf{20/09/2017}. Quem não entregar até esse dia irei fazer uma observação em sua avaliação final.

\subparagraph{Observação 2:} Ao escrever a bibliografica, por favor, coloquem seguindo o modelo abaixo (facilita e muito quando for convetido para latex):\\
\\
  title={The texbook},\\
  author={Knuth, D.E. and Bibby, D. and Makai, I.},\\
  volume={1993},\\
  year={1986},\\
  publisher={Addison-Wesley}

 
\section{Requisitos de eletrônica}
\subsection{Módulo controlador - sistema principal}
\paragraph{} O módulo controlador do sistema principal consiste em um módulo que irá controlar o robô eletromecânico.
\paragraph{} Os integrantes devem pesquisar sobre o sistema que controla o robô, esteja conectado a um computador ou notebook, enviando os  sequintes comandos de movimento (considere a superfície de contato placa):
\subparagraph{$\rightarrow$} Translação (cilíndrico) em relação ao eixo inicial do braço mecânico
\subparagraph{$\rightarrow$} Nível, ou seja, mais alto ou mais baixo (aplicar mais pressão ou ajustar a altura para o exame médico)
\subparagraph{$\rightarrow$} Esférico, ou seja, esse será o movimento de onde estará acoplado o sensor de ultrasom

\paragraph{Integrantes:}
\begin{itemize}
\item Vitor Carvalho
\item Matheus Carvalho
\item Vitor Rangel
\end{itemize}

\subsection{Módulo de transmissão e recebimento de dados de movimento e imagem - transição entre os sistemas}
\paragraph{} Neste módulo teremos como será transmitido os sinais advindos do movimento do robô, enviado pelo módulo controlador, e o recebimento dos dados de imagem do sensor de ultrasom. 

\paragraph{} Os integrantes devem pesquisar métodos de transmissão de dados (hardware) que possam transmitir via internet à cabo, wifi e gsm (4g). Procurem também qual seria a confiabilidade de cada um desses métodos, delays, erros de transmissão e recebimento. Além de como ele iria se conectar a um notebook e como seria programado (linguagem apenas)

\paragraph{Integrantes:}
\begin{itemize}
\item Pedro Lucas
\item Anderson Sales
\item Harolso Júnior
\item Bruno Alves
\end{itemize}

\subsection{Módulo sensor ultrasom - sistema escravo}

\paragraph{} Esse módulo, como o próprio nome diz, busca verificar no mercado os sensores de ultrasom disponíveis e como ocorre o seu funcionamento. 

\paragraph{} Há ideia é que os integrantes pesquisem os sensores de ultrasom existentes no mercado, indicando:
\subparagraph{$\rightarrow$} Funcionamento;
\subparagraph{$\rightarrow$} Preço; 
\subparagraph{$\rightarrow$} Quais tipos de imagem ele transmite; 
\subparagraph{$\rightarrow$} Taxa de atualização;
 resolução
\subparagraph{$\rightarrow$} Todos os tipos de sinais que ele transmite; 
\subparagraph{$\rightarrow$} Especificações energéticas e de tempetura que pode ser trabalhada;
\subparagraph{$\rightarrow$} Quais são os tipos de sinais (entradas no sensor em si) para ativação das suas várias funcionalidades.

\paragraph{}Indicar qual seria a vantagem e desvantagens entre os sensores pesquisados. Não se esqueçam, nós não iremos criar do zero um sensor de ultrasom, iremos utilizar os já existentes no mercado.

\paragraph{Integrantes:}
\begin{itemize}
\item Pedro Henrique
\item Tiago Pereira (Gerente)
\item Marcos Lima
\end{itemize}

\subsection{Módulo Eletrônico para o braço mecânico - sistema escravo}

\paragraph{} Neste módulo temos como objetivo fazer toda a pesquisa por trás da utilização de motores de passo, atuadores, circuito geral eletrônico de controle, ..., para o braço mecâtrônico que movimentará o sensor de ultrasom

\paragraph{} Os integrantes devem pesquisar is sequintes tópicos: 
\subparagraph{$\rightarrow$} Atuadores, motores de passo, servomotores (funcionamento, melhor aplicabilidade, comparativo);
\subparagraph{$\rightarrow$} "Robô manipulador esférico" (onde será encaixado o sensor de ultrasom) e seu funcionamento; 
\subparagraph{$\rightarrow$} Métodos para controlar os motores, como fpgas, arduino, etc. Eles receber os sinais de movimento advindos do módulo controlador, processar e transmitir para cada um dos dispositivos de movimento (motores);

\paragraph{} Façam uma pesquisa voltada para um nível onde pode ser implementado, a ideia é que o protótipo seja o braço mecânico se movimentando a partir de um controle remoto;

\paragraph{Integrantes:}
\begin{itemize}
\item Filipe de souza 
\item Marcelo Magalhaes
\item Vinicius Z de Oliveira
\item Bruno Souza
\end{itemize}

\end{document}